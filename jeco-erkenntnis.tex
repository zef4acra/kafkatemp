% specify 12pt font in the settings for your document class
\documentclass[12pt]{article}

% set margins 
\usepackage[top=4cm, bottom=4cm, left=3cm, right=2.5cm]{geometry}

% make entire document double spaced
\usepackage{setspace}
\doublespacing

% ensure footnotes are full sized and double spaced
\usepackage{footmisc}
\renewcommand{\footnotelayout}{\doublespacing\normalsize}

% remove page numbers
% \usepackage{nopageno}

% left justify text
% \makeatletter
% \newcommand\iraggedright{%
%  \let\\\@centercr\@rightskip\@flushglue \rightskip\@rightskip
% \leftskip\z@skip}
% \makeatother
% \iraggedright

%%%%%%%%%%%%%%%%%%%%%%%%%%%%%%%%%%%%%%%%%%%%%%%%%%%%%%%%%%%%%%%%%%%%%%%%%%%%%%%%%%%%%%%%%%%%%%

\usepackage{october}
\usepackage{lineno}
\newtheorem{theorem}{Theorem}[section]
\newtheorem{corollary}{Corollary}[theorem]
\newtheorem{lemma}[theorem]{Lemma}
\newtheorem{definition}{Definition}[section]

% For BJPS
% \hyphenpenalty=10000
% \hbadness=10000

\begin{document}
% \setpagewiselinenumbers
% \modulolinenumbers[5]
% \linenumbers
% For BJPS
% \raggedright
% \doublespacing

% z=16;pdftk A=../apa-central-2017.pdf cat A1-$z output ../wc-out.pdf

\title{Asymmetry and the Geometry of Reason}
\author{for blind review}
\date{}
\maketitle
% \newcounter{expls}
% \doublespacing

\maketitle

\begin{abstract}
  {\noindent}The geometry of reason is the view that the underlying
  topology for credence functions is a metric space, on the basis of
  which axioms and theorems of epistemic utility for partial beliefs
  are formulated. It implies that Jeffrey conditioning must cede to an
  alternative form of conditioning. The latter fails a long list of
  plausible expectations. One solution to this problem is to reject
  the geometry of reason and accept information theory in its stead.
  Information theory comes fully equipped with an axiomatic approach
  which covers probabilism, standard conditioning, and Jeffrey
  conditioning. It is not based on an underlying topology of a metric
  space, but uses a non-commutative divergence instead of a symmetric
  distance measure. I show that information theory, despite initial
  promise, also fails to accommodate basic epistemic intuitions.
\end{abstract}

\section{Introduction}
\label{intr}

There\rcut{1} are various ways in which epistemic norms for partial
beliefs are justified. Three important standards of justification in
the literature are pragmatics, accuracy, and evidence. A partial
belief, as opposed to a full belief, expresses uncertainty whether or
not a proposition is true. In formal epistemology, this uncertainty is
captured in mathematical models. Examples for epistemic norms are
probabilism (partial beliefs are numerically most effectively
represented as probabilities), Bayesian conditionalization (a rational
agent updates beliefs using conditional probabilities if the updating
scenario permits it), the principle of indifference (if there is no
further information, mutually disjoint and jointly exhaustive events
are equiprobable), and additional updating methods beyond standard
conditioning (Jeffrey conditioning, affine constraints). Whether the
justification for (or rejection of) these norms cites pragmatic,
alethic, or evidential reasons, one important ingredient of the formal
model is scoring rules.

This paper investigates scoring rules for partial beliefs that
exclusively reward and penalize on epistemic grounds. This aligns my
paper roughly with the school that privileges alethic justification
for epistemic norms, but it should be noted that the debate over
scoring rules also has important implications for pragmatists and
evidentialists, who hold that epistemic norms are rooted in decision
theory and the appropriate relationship between beliefs and the
evidence on which they are based, respectively.

\begin{quotex}
  \beispiel{Trichotomy}\label{ex:iexahmah} A game between the home
  team and the away team ends in a win (for the home team), a loss, or
  a tie.
\end{quotex}

I am restricting myself to finite algebras of propositions, as in
{\xample}~\ref{ex:iexahmah}, where exactly one of three random
outcomes takes place. Let these outcomes (or possible worlds) be
$\xi_{1},\xi_{2},\xi_{3}$. Let the agent's report over these three
outcomes be $c=(c_{1},c_{2},c_{3})^{\intercal}$ (the transpose
$^{\intercal}$ symbol merely turns the list of numbers into a vector).
Another restriction for this paper shall be that the $c_{i}$ are
non-negative real numbers so that the vector $c$ is located in the
non-negative orthant $\mathcal{D}_{0}$ of an $n=3$ dimensional vector
space. The agent is penalized for reporting $c$ according to a loss
function which she wants to minimize (she has no other concerns). Her
penalty is
\begin{equation}
  \label{eq:choirail}
  S(\xi_{i},c)
\end{equation}
once it is established that $\xi_{i}$ is the realized outcome. To
discourage any dishonesty on her part, a common requirement is that
the scoring rule be proper. The propriety of a scoring rule ensures
that an agent reports the same distribution according to which she
thinks the random process selects the outcome. A scoring rule $S$ is
strictly proper if and only if
\begin{equation}
  \label{eq:xioputhu}
  % \sum_{i=1}^{n}c_{i}S(\xi_{i},c)=\min_{\hat{c}\in\mathcal{D}_{0}}\left\{\sum_{i=1}^{n}c_{i}S(\xi_{i},\hat{c})\right\}
  \sum_{i=1}^{n}c_{i}S(\xi_{i},c)<\sum_{i=1}^{n}c_{i}S(\xi_{i},\hat{c})\mbox{ for all }\hat{c}\in\mathcal{D}_{0}\setminus\{c\}
\end{equation}
A scoring rule $S$ is proper if and only if (\ref{eq:xioputhu}) is
true with a $\leq$ symbol rather than $<$. In the following, I will
consistently say \qnull{proper} and \qnull{propriety} in abbreviation
for \qnull{strictly proper} and \qnull{strict propriety.} Propriety
guarantees that an agent is motivated to report the distribution that
they deem to be the one according to which the random outcomes are
generated. Propriety significantly narrows down the set of acceptable
scoring rules. John McCarthy showed in a seminal paper that propriety
requires the existence of a convex entropy function, for which the
scoring rule is a type of derivative (see \scite{7}{mccarthy56}{}).

The problem I am addressing in this paper is whether there are further
restrictions on rationally acceptable scoring rules. McCarthy's
theorem leaves open the possibility for symmetric and non-symmetric
scoring rules. A symmetric scoring rule assigns as much loss to a
reported credence $c$ when the true distribution is $\bar{c}$ as it
does when the report is $\bar{c}$ and the true distribution is $c$.
Richard Pettigrew has recently defended symmetric scoring rules (for
example in \scite{8}{pettigrew16}{80}).

The claim at the heart of my paper is that a defence of symmetry
reveals a more fundamental misapprehension about partial beliefs and
their relationships to each other. The misapprehension is that there
is a geometry of partial beliefs which can be visualized. It is
tempting to view a credence $c=(c_{1},c_{2},c_{3})^{\intercal}$, for
example, as a vector in 3-dimensional space and then evaluate its
distance to other credences in terms of its metric distance to them. I
will call this view, following Hannes Leitgeb and Richard Pettigrew
(see \scite{8}{leitgebpettigrew10i}{210}), the geometry of reason.

Thomas Mormann explicitly warns against the assumption that the
metrics for a geometry of logic is Euclidean by default: \qeins{All
  too often, we rely on geometric intuitions that are determined by
  Euclidean prejudices. The geometry of logic, however, does not fit
  the standard Euclidean metrical framework} (see
\scite{8}{mormann05}{433}; also \scite{7}{miller84}{}). Mormann
concludes in his article \qeins{Geometry of Logic and Truth
  Approximation,}

\begin{quotex}
  Logical structures come along with ready-made geometric structures
  that can be used for matters of truth approximation. Admittedly,
  these geometric structures differ from those we are accustomed
  with, namely, Euclidean ones. Hence, the geometry of logic is not
  Euclidean geometry. This result should not come as a big surprise.
  There is no reason to assume that the conceptual spaces we use for
  representing our theories and their relations have a Euclidean
  structure. On the contrary, this would appear to be an improbable
  coincidence. \scite{3}{mormann05}{453}
\end{quotex}

For Pettigrew, the geometry of reason stands out as an appealing
account because its associated scoring rule, the Brier score, is (up
to linear transformations) unique once symmetry is required. Pettigrew
even has an argument why this uniqueness in its own right has
appealing features. I will provide a more detailed summary of the
various requirements that one might have with respect to scoring
rules. An alternative to the geometry of reason emerges from this
analysis, which I will call information theory.

Information theory, just like the geometry of reason, has an
associated scoring rule: the Log score. The two different scoring
rules, Brier score and Log score, license different updating methods
in dynamic partial belief theory. They agree on Bayesian updating in
the standard case (using conditional probabilities), but they disagree
on updating in a Jeffrey-type updating scenario. Both scoring rules
agree on recommending probabilism. I will argue that information
theory is the better Bayesian, because Bayesian standard conditioning
is smoothly generalized in information theory to more general updating
situations (the technical term is affine constraints, of which
Jeffrey-type updating scenarios are a special case). The
generalization for the geometry of reason is bumpy at best,
implausible at worst. I will present this view in full detail in the
main body of the paper.

The Log score is asymmetric, but unlike the Brier score not unique
among its asymmetric peers. According to Pettigrew's independent
argument why it is a good idea to have a unique scoring rule this
would count against the Log score and against information theory. The
Log score, however, is unique in fulfilling a locality requirement
that arguably commands as much plausibility as symmetry. Yet the tenor
of my paper is that Pettigrew's independent argument for uniqueness is
suspect (in defence of Pettigrew, many of the claims in his book
\emph{Accuracy and the Laws of Credence} do not depend on it) and that
neither the Brier score's symmetry nor the Log score's locality is
sufficient to make them uniquely superior to other scoring rules.

I believe that there are serious problems with the geometry of reason,
to the point where I would reject it as a plausible formal account of
partial beliefs. As I will show, however, there are serious problems
with information theory and how it accommodates epistemic intuitions
as well. These are not insurmountable. My hope is that a further
detachment from geometry can give us a better understanding of why
information theory has the odd features that I will highlight in the
paper.

\section{Features of Scoring Rules}
\label{sec:vidiedoo}

\subsection{List of Features and Preliminaries}
\label{subsec:ayudoosa}

Consider the following list of features for a scoring rule SR.
\begin{description}
\item[propriety] The SR encourages an agent to report the
  distribution which is her best guess for what generates the random
  event.
\item[geometry] The divergence function associated with the SR
  is a metric. Consequently, credence functions can be
  \qnull{visualized} with a distance defined between them.
\item[information] The entropy function associated with the SR
  fulfills Shannon's axioms for an entropy function. 
\item[symmetry] The expected penalty for a reported distribution with
  respect to the true distribution is the same when the roles of the
  reported and true distribution are reversed.
\item[locality] How a distribution scores when an event takes place
  depends only on the credence assigned by the distribution to this
  event.
\item[horizon] The divergence function associated with the SR has a
  tendency to measure distributions near the centre as being closer
  together than distributions near the extremes, all else being equal.
\item[conditioning] The SR licenses standard conditioning.
\item[reductio-resistance] The SR is not led ad absurdum by licensing
  implausible updating methods.
\item[univocal dominance] The SR is uniquely superior to all other SRs
  in order to address the Bronfman objection.
\end{description}

Here is a brief summary with evaluative annotation (plausibility, for
example, means that in conclusion to my arguments in this paper I find
the requirement plausible). If my evaluation is endorsed, only the Log
score qualifies as a rationally acceptable scoring rule and
information theory (as opposed to the geometry of reason) is
vindicated; however, the violation of \textsc{horizon} must be
explained, which is beyond the purview of this paper.

\medskip

\begin{tabular}{|l|l|l|l|l|}\hline
  \textbf{property} & \textbf{Brier score} & \textbf{Log score} & \textbf{other scores} & \textbf{annotation} \\ \hline
  propriety & yes & yes & yes & commonly accepted \\ \hline
  geometry & yes & no & no & implausible \\ \hline
  information & no & yes & no & plausible \\ \hline
  symmetry & yes & no & no & implausible \\ \hline
  locality & no & yes & no & weakly plausible \\ \hline
  horizon & no & long story & perhaps & plausible \\ \hline
  conditioning & yes & yes & perhaps & plausible \\ \hline
  reductio-resistance & no & yes & perhaps & plausible \\ \hline
  univocal dominance & yes & yes & perhaps & implausible \\ \hline
\end{tabular}

\medskip

In the following, let $\mathcal{A}$ be the comprehensive algebra over
a finite set of events $\Omega$. This means that all possible
combinations of events in terms of negation, union, and intersection
are in $\mathcal{A}$. If $\Omega=\{A,B\}$, for example, then
$\mathcal{A}$ contains all possible combinations of
\begin{equation}
  \label{eq:ietoobai}
  \bigcup_{i=1}^{4}\omega_{i}^{j(i)}
\end{equation}
where 
\begin{equation}
  \label{eq:ongeefoo}
  \omega_{1}=\urcorner{}A\cap\urcorner{}B\hspace{.5in}
  \omega_{2}=\urcorner{}A\cap{}B \hspace{.5in}
  \omega_{3}=A\cap\urcorner{}B \hspace{.5in}
  \omega_{4}=A\cap{}B
\end{equation}
and $j(i)$ either negates $\omega_{i}$ or not. If the cardinality of
$\Omega$ is $m$, then the cardinality of $\mathcal{A}$ is $n=2^{{2}^{m}}$.
For our example $\Omega=\{A,B\}$, $\mathcal{A}$ contains 16 elements;
any credence function can therefore be viewed as a vector in the
positive orthant of $\mathbb{R}^{16}$. An orthant is a generalization
of a quadrant in $\mathbb{R}^{2}$. I will write $\mathcal{D}_{0}$ if
the orthant includes vectors which have elements that equal zero;
$\mathcal{D}$ if all elements of the vector are greater than zero. The
zero vector itself is not an element of $\mathcal{D}_{0}$. Requiring
that credence functions are finite, or non-negative, or positive in
all elements, or regular in other ways is artificial in order to aid
discussion. Examinations of what happens when these regularity
conditions are weakened are always welcome.

Probability functions are a strict subset of credence functions. In
the vector space $\mathbb{R}^{n}$ they form a $(2^{m}-1)$-dimensional
simplex. For two events, it is a three-dimensional tetrahedron in
$\mathbb{R}^{16}$;\tbd{really?} for three events, it is a 7-simplex in
$\mathbb{R}^{256}$. The reason why probability functions are of much
lesser dimension than general credence functions is because they are
constrained by logical entailment relationships and Kolmogorov's
axioms. In the following, I will ignore the logical entailment
relationships. This means that I will assume that credence functions
obey these entailment relationships as probability functions do. This
means that I can concentrate on a case such as the trichotomoy in
{\xample}~\ref{ex:iexahmah} and restrict my attention to credence
functions in $n$-dimensional space. All arguments defending
probabilism in this paper do not only justify probabilistic credence
functions over non-probabilistic ones that obey the logical entailment
relationships, but ad fortiorem also over non-probabilistic ones that
disobey the logical entailment relationships.

Possible worlds $\xi_{i}$ corresponding to $\omega_{i}$ are
not credence functions, but they can be embedded by defining the
vector elements $\xi_{k}\in\{0,1\}$ depending on whether $\omega_{i}$
is (1) or is not (0) negated in (\ref{eq:ietoobai}), $k=1,{\ldots},n$.
This is somewhat artificial, especially the choice of the number 1,
and any account of epistemic norms must prove itself to be robust if
this number is changed to something else that makes sense (see
\scite{8}{howson08}{20}; for a response \scite{7}{pettigrew16}{}, part
I, chapter 6).

\subsection{Scoring Rules, Entropy, Divergence}
\label{subsec:weingoov}

Bruno de Finetti showed that the probability functions form the convex
hull of possible worlds so embedded (see de Finetti,
\scite{11}{definetti17}{}, subsection 3.4). It is relatively
straightforward to see that for any vector $c$ in the vector space of
credence functions, there is a vector $p$ in the convex hull of
probability functions which is closer to all possible worlds than $c$.
If $c$ is not an element of the convex hull, then the vector $p$ is
strictly closer to all possible worlds.

There are two important points to make about de Finetti's theorem. (1)
It shows in what sense probability functions are privileged over other
credence functions. This can be cashed out in terms of pragmatics (see
\scite{7}{savage71}{}) or accuracy (see \scite{7}{joyce98}{}) (perhaps
also in terms of evidence; I am not familiar with the literature). (2)
There is a need to define what it means for one credence function to
be close to another credence function. One could simply use metric
distance. I am hoping to make the case in what follows that this is
implausible.

Let us start more modestly with a scoring rule and restrict ourselves
to probability functions $\mathcal{P}\subset\mathcal{D}_{0}$.
Probabilism, after all, is not where the geometry of reason and
information theory disagree.
% Options for how (symmetric) distance or (non-symmetric) divergence is
% most effectively measured will emerge. By effectiveness I mean
% justificatory force for normative epistemic arguments.
I have defined scoring rules in {\quation}~(\ref{eq:choirail}) and the
associated propriety in {\quation}~(\ref{eq:xioputhu}). McCarthy
characterizes proper scoring rules in a theorem whose proof he omits
(see \scite{8}{mccarthy56}{654}). Thankfully, Arlo Hendrickson and
Robert Buehler provide a proof (see
\scite{8}{hendricksonbuehler71}{1918}).

\begin{definition}
  \label{def:yahjoith}
  Let $\bar{D}$ be a convex subset of $\mathcal{D}$, $H$ be a function
  $H:\bar{D}\rightarrow\mathbb{R}$, and $q,q^{\ast}\in{}\bar{D}$ such that
  \begin{equation}
    \label{eq:fooceiya}
    H(p)\leq\langle{}p-q,q^{\ast}\rangle+H(q)\mbox{ for all }p\in{}\bar{D}.
  \end{equation}
Then $q^{\ast}$ is a supergradient of $H$ at $q$ relative to $\bar{D}$. 
\end{definition}

$\langle{}.,.\rangle$ is the inner product of two vectors (or the
matrix product if dual spaces are used). The supergradient is the
gradient wherever the function is differentiable.

\begin{definition}
  \label{def:ahthaive}
  A function $f:V\subset\mathbb{R}^{k}\rightarrow\mathbb{R}$ is
  homogeneous of degree $d$ if and only if
  \begin{equation}
    \label{eq:ooveighe}
    f(\alpha{}x)=\alpha^{d}f(x)\mbox{ for all }\alpha>0.
  \end{equation}
\end{definition}

Use of Euler's Homogeneous Function Theorem (see
\scite{7}{zill11}{718}) allows for the proof of McCarthy's Theorem.
Let $\nabla{}H$ be the gradient of the function $H$ if it exists, i.e.
\begin{equation}
  \label{eq:ailaekoi}
  \nabla{}H(x)=\left(\frac{\partial{}H}{\partial{}x_{1}}(x),{\ldots},\frac{\partial{}H}{\partial{}x_{n}}(x)\right)^{\intercal}
\end{equation}
and
\begin{equation}
  \label{eq:ahphaete}
  S(x)=\left(S(\xi_{1},x),{\ldots},S(\xi_{n},x)\right)^{\intercal},\Xi=\{\xi_{1},{\ldots},\xi_{n}\}
\end{equation}

\begin{theorem}[McCarthy's Theorem] 
  \label{thm:lahpoodu}
  A scoring rule
  $S:\Xi\times\mathcal{P}\rightarrow\mathbb{R}\cup\{-\infty,\infty\}$
  is proper if and only if there exists a function
  $H:\mathcal{D}_{0}\rightarrow\mathbb{R}$ which is (a) homogeneous of
  the first degree, (b) convex,\tbd{you may have to change everything
    to rewards rather than loss in order to maintain convexity rather
    than concavity here} and (c) such that $S$ is a supergradient of
  $H$ relative to $\mathcal{D}_{0}$ at $p$ for all $p\in\mathcal{P}$.
\end{theorem}

If $H$ is differentiable, then $\nabla{}H(p)=S(p)$. It is important to
take the partial derivative of $H$ as a function defined on
$\mathcal{D}_{0}$, not just as a function defined on $\mathcal{P}$. As
Hendrickson and Buehler point out, this is the error in
\scite{8}{marschak59}{97}, where the Log score appears to be a
counterexample to McCarthy's theorem. None of this is new, but it
gives us leverage for what is to follow. Not only is McCarthy's
Theorem a powerful characterization theorem for proper scoring rules,
it also associates an entropy function $H$ and a divergence $D$ with
each scoring rule. With McCarthy's result in hand, scoring rules now
come as triplets of scoring rules, entropy functions, and divergences.

\begin{definition}
  \label{def:logahpoh}
  The entropy $H$ associated with $S$ is defined as per McCarthy's
  Theorem (see {\heorem}~\ref{thm:lahpoodu}); the divergence associated
  with $S$ is defined to be
\begin{equation}
  \label{eq:reeshooz}
  D_{S}(c\|\hat{c})=H(c)-H(\hat{c})-\langle{}c-\hat{c},S(\hat{c})\rangle
\end{equation}
\end{definition}

Here are two example, the Log score and the Brier score. All summation
indices go from 1 to $n$. If $x\in\mathcal{P}$, then
$\sum_{k}x_{k}=1$. The loss function or scoring rule for the Log score
is
\begin{equation}
  \label{eq:oneeraiw}
  \mbox{(LS) }S(\xi_{i},x)=\left(\ln\sum_{k}x_{k}\right)-\ln{}x_{i}
\end{equation}
For the Brier score, it is
\begin{equation}
  \label{eq:hiexaeji}
  \mbox{(BS) }S(\xi_{i},x)=\frac{2x_{i}}{\sum_{k}x_{k}}-1-\sum_{j}\left(\frac{x_{j}}{\sum_{k}x_{k}}\right)^{2}
\end{equation}
The corresponding entropy functions are
\begin{equation}
  \label{eq:yeuthohn}
  \mbox{(LS) }H(x)=-\sum_{i}{}x_{i}\ln\frac{x_{i}}{\sum_{k}x_{k}}
\end{equation}
\begin{equation}
  \label{eq:ahfooyai}
  \mbox{(BS) }H(x)=\sum_{i}{}x_{i}\left(\frac{2x_{i}}{\sum_{k}x_{k}}-1-\sum_{j}\left(\frac{x_{j}}{\sum_{k}x_{k}}\right)^2\right)
\end{equation}
The reader can verify that
\begin{equation}
  \label{eq:eikughoh}
  \nabla{}H(x)=S(x)
\end{equation}
for both the Log score and the Brier score. This is where we need the
entropy to be defined on $\mathcal{D}\supset\mathcal{P}$ in order to
avoid Marschak's error from above. For the Log score, the divergence
is the Kullback-Leibler divergence,
\begin{equation}
  \label{eq:engeequu}
  D_{\mbox{\tiny LS}}(p\|q)=\sum_{i}p_{i}\ln\frac{p_{i}}{\sum_{i}p_{k}}-\sum_{i}p_{i}\ln\frac{q_{i}}{\sum_{k}q_{k}}
\end{equation}
For the Brier score, the divergence is 
\begin{equation}
  \label{eq:aphooghi}
  D_{\mbox{\tiny BS}}(p\|q)=\sum_{i}p_{i}\left[\sum_{j}\left(\frac{q_{j}}{\sum_{k}q_{k}}-\delta_{ij}\right)^{2}-\sum_{j}\left(\frac{p_{j}}{\sum_{k}p_{k}}-\delta_{ij}\right)^{2}\right]
\end{equation}
where $\delta_{ij}$ is the Kronecker delta.

The Brier score divergence is symmetric for probability functions and
thus can operate as a distance function on them. For both Log score
and Brier score, the divergence is the difference between the expected
loss of reported $q$ when $p$ is the true distribution and the entropy
of $p$, which is the expected loss of reported $p$ when $p$ is the
true distribution.

In section\tbd{reference} I will show using convex conjugates that
only the Brier score (and its close relatives) fulfill
\textsc{symmetry}. In section\tbd{reference} I will show that only the
Log score fulfills \textsc{locality}. These are not new results,
although the proof using convex conjugates is original. Once these
results are established, I have the tools to address Pettigrew's
argument for \textsc{univocal dominance}. The core pieces of the paper
are sections\tbd{reference}, where I show that the Brier score
violates \textsc{reductio-resistance} and both Brier score and Log
score violate \textsc{horizon}. 

\section{Symmetry}
\label{sec:oolohdej}

The Brier score and its close relatives (linear transformations, which
only differ from the Brier score in the sense that they bill/pay in a
different currency and provide a different initial penalty/reward) are
the only proper scoring rules fulfilling \textsc{symmetry}.

\section{Locality}
\label{sec:siethohj}

\subsection{Rewarding Uncertainty About Non-Realized Outcomes}
\label{subsec:paivaete}

The Brier score, the Spherical score (another proper scoring rule),
and many other scoring rules depend on all components of the vector
$p$ representing a probabilistic credence function. A scoring rule
fulfilling the \textsc{locality} requirement only depends on the
probability assigned to the event that is the realized outcome (for a
characterization of local scoring rules that are local in a less
restrictive sense see \scite{7}{dawidetal12}{}). In
{\ubsection}~\ref{subsec:kaixitun} I show, using Leonard J. Savage's
proof (see \scite{8}{savage71}{794}), that the only scoring rule
fulfilling \textsc{locality} are the Log score and its not relevantly
different close relatives.

\begin{quotex}
  \beispiel{Tokens}\label{ex:augheebi} Casey draws from a bag with $n$
  kinds of tokens in it: colour 1, colour 2, {\ldots}, colour $n$.
  Tatum reports the forecast $(p_{1},{\ldots},p_{n})$. Tatum's
  forecast agrees with the axioms of probability.
\end{quotex}

Let $p_{1}$ be fixed and colour 1 be the realized outcome. If the
Brier score is used, Tatum's penalty $T$ depends on
$p_{2},{\ldots},p_{n-1}$ and is
\begin{equation}
  \label{eq:kaewaeho}
  T(p_{2},{\ldots},p_{n-1})=1-2p_{1}+\sum_{i=1}^{n-1}p_{i}^{2}+\left(1-\sum_{i=1}^{n-1}p_{i}\right)^{2}
\end{equation}
$T$ reaches its minimum where $p_{i}=\frac{1-p_{1}}{n-1}$ for
$i=2,{\ldots},n-1$. The higher the entropy of Tatum's non-realized
probabilities, the less stinging Tatum's penalty. The Brier score thus
penalizes Tatum (1) for not correctly identifying colour 1 as the
realized outcome, but also (2) for reporting variation in the
non-realized probabilities. Even though this is the Brier score, it
has a ring of information theory to it. The Log score depends only on
the realized probability. I.J. Good appears to have favoured such a
scoring rule (see \scite{8}{good52}{112}).

Because I feel the intuitive appeal of information theory, I consider
\textsc{locality} to be only weakly plausible. There is a sense in
which you may want to reward a forecaster not only for assigning a
high probability to the realized outcome, but also for uncertainty
about the outcomes that were not realized. Of course, doing so
sometimes results in a greater loss for Tatum than for Casey even if
Tatum assigned a higher probability to the realized outcome. As an
example, let Tatum's forecast be $(0.12,0.86,0.02)$ and Casey's be
$(0.10,0.54,0.36)$. Even though Tatum assigned $12\%$ to colour 1
while Casey assigned $10\%$, and Casey drew a token of colour 1, Tatum
is penalized more severely at $1.5144$ compared to Casey at $1.2312$
using the Brier score.

\subsection{Bronfman Objection}
\label{subsec:eemeingo}

Here is how \textsc{locality} may still work in favour of information
theory against the geometry of reason. To set the scene, I should
mention that there are at least two publication anomalies in the study
of scoring rules. I already mentioned the earlier one: McCarthy
omitted the proof to one of its most important theorems. The proof,
using Euler's Theorem, is not trivial and was published almost twenty
years later by Hendrickson and Buehler. The later publication anomaly
is that Aaron Bronfman wrote an excellent article about a problem with
using supervaluationist semantics to justify probabilism (see
\scite{7}{bronfman09}{}). Then he decided not to publish it. The
manuscript has circulated and is available online.

It is called \qeins{A Gap in Joyce's Argument for Probabilism.} Jim
Joyce provides a non-pragmatic (i.e.\ alethic) vindication of
probabilism by demonstrating that given a particular proper scoring
rule, any non-probabilistic credence $c$ is dominated by a
probabilistic credence function $p$ (depending on $c$) in the sense
that $p$ is strictly closer to all possible worlds (and therefore more
accurate) than $c$. No probabilistic credence function is dominated in
this way (see \scite{7}{joyce98}{}). The proof is a version of de
Finetti's theorem referred to in {\ubsection}~\ref{subsec:weingoov}.

I owe the following characterization of Bronfman's objection to
Pettigrew (see chapter 5 in \scite{7}{pettigrew16}{}; for the Pater
Peperium case see \scite{7}{paul16}{}).

\begin{quotex}
  \beispiel{Pater Peperium}\label{ex:quuyaree} I must choose between
  three sandwich options: Marmite, cheese, and Pater Peperium (or
  Gentleman's Relish).
\end{quotex}

I have eaten cheese sandwiches before and feel indifferent about them.
I have never had a Marmite or Pater Peperium sandwich, but know that
people either love Marmite and hate Pater Peperium or vice versa.
There appears to be nothing irrational about choosing the cheese
sandwich even though either way (whether I am of the
love-marmite-hate-pater-peperium or hate-marmite-love-pater-peperium
type) there is a better sandwich to choose. 

Joyce has shown that for any proper scoring rule, a non-probabilistic
credence function is accuracy dominated by a probabilistic credence
function. The Bronfman objection is that you can show that there is
always another proper scoring rule (Bronfman shows that only having
two candidate quadratic loss scoring rules suffices to make this
point) by which moving from the accuracy dominated credence function
to the probabilistic credence function results in a loss at some
possible world. Unless we settle on a unique scoring rule to do the
accounting, Joyce's non-pragmatic vindication of probabilism is
undermined. 

Pettigrew uses Bronfman's objection to propose \textsc{univocal
  dominance}. It is an appealing feature of a scoring rule to have
some claim to uniqueness in order to address Bronfman's objection. The
Brier score has this claim: it is (up to linear transformation, which
do not make a relevant difference) the only proper scoring rule which
fulfills \textsc{symmetry}. Unfortunately for Pettigrew, the Brier
score is not the only score with this kind of claim. The Log score is
the only proper scoring rule which fulfills \textsc{locality}. We
could now haggle over which uniqueness claim is stronger. In some
ways, this paper is meant to undermine the intuitive appeal of
\textsc{symmetry} altogether. I will not, however, push
\textsc{univocal dominance} and \textsc{locality} as joint
justification for the Log score, as Pettigrew pushes \textsc{univocal
  dominance} and \textsc{symmetry} as joint justification for the
Brier score. 

Pettigrew's argument is suspect (he by no means is unaware of its
tenuous appeal and reiterates that many of his results stand even if
\textsc{univocal dominance} is implausible). Let a uniqueness claim
have dependent and independent reasons. The dependent reasons justify
the uniqueness on account of the features that the object of the
uniqueness claim exhibits. The independent reasons make no reference
to these features, but provide a reason to have a successful candidate
for winning the uniqueness contest. I do not see how these independent
reasons add to the epistemic justification for the uniqueness claim.

\subsection{Proof of Locality Uniqueness for the Log Score}
\label{subsec:kaixitun}

Let there be a function $f:(a,b)\rightarrow\mathbb{R}$
($a,b\in\mathbb{R}$ with $a<b$) for which the following is true: for
every $x\in(a,b)$ there exists a linear function
$L_{x}:\mathbb{R}\rightarrow\mathbb{R}$ such that $L_{x}(x)=x$ and
\begin{equation}
  \label{eq:jeedushe}
  L_{x}(y)<f(y)\mbox{ for all }y\in(a,b)
\end{equation}
The conditions basically say that $f$ has a subgradient at every point
in its domain. Then the function is strictly convex, i.e.
\begin{equation}
  \label{eq:poawaimo}
  f(\lambda{}y+(1-\lambda)\hat{y})<\lambda{}f(y)+(1-\lambda)f(\hat{y})
\end{equation}
for all $0<\lambda<1$ and for all $y,\hat{y}$ in the domain of $f$. A
theorem of convex analysis tells us that such a function is almost
everywhere differentiable (i.e.\ the set of points where it is not
differentiable is countable; see {\heorem}~25.5 in
\scite{7}{rockafellar97}{}). 

Define $f_{i}(p_{i})=S(\xi_{i},p_{i})$ for a scoring rule fulfilling
\textsc{locality} and \textsc{propriety}. $f_{i}$ are functions from
$[0,1]\rightarrow\mathbb{R}\cup\{-\infty,\infty\},i=1,{\ldots},n$.
Propriety requires
\begin{equation}
  \label{eq:eushunie}
  \sum_{i=1}^{n}p_{i}f_{i}(p_{i})>\sum_{i=1}^{n}q_{i}f_{i}(p_{i})\mbox{ for all }p,q\in\mathcal{P}
\end{equation}
where $p=(p_{1},{\ldots},p_{n})^{\intercal}$ and
$q=(q_{1},{\ldots},q_{n})^{\intercal}$. Let $0<k<1$ be arbitrary, but
fixed, and define
\begin{equation}
  \label{eq:rozosuzo}
  g_{k}(x)=xf_{1}(x\cdot{}k)+(1-x)f_{2}((1-x)\cdot{}k)
\end{equation}
Let $\alpha,\beta\in(0,1)$ with $\alpha\neq{}\beta$.
(\ref{eq:eushunie}) for the two distributions
$p=(\alpha\cdot{}k,(1-\alpha)\cdot{}k,\frac{1-k}{n-2},{\ldots},\frac{1-k}{n-2})$
and
$q=(\beta\cdot{}k,(1-\beta)\cdot{}k,\frac{1-k}{n-2},{\ldots},\frac{1-k}{n-2})$
implies
\begin{equation}
  \label{eq:luophaek}
  g_{k}(\alpha)>(\alpha\cdot{}k)f_{1}(\beta\cdot{}k)+((1-\alpha)\cdot{}k)f_{2}((1-\beta)\cdot{}k)
\end{equation}


% \section{References}
% \label{refs}

% \nocite{*} 

% set bibtex bibliography to chicago style
% \bibliographystyle{chicago}

\bibliographystyle{ChicagoReedweb} 
\bibliography{bib-2902}

\end{document}
